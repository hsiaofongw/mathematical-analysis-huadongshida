\documentclass{ctexart}
\usepackage[
    a4paper,
    top=0.8in,
    left=0.8in,
    right=0.8in,
    bottom=1in
]{geometry}
\usepackage{amsmath}
\usepackage{amssymb}
\usepackage{float}
\usepackage{booktabs}
\usepackage{amsthm}
\usepackage{float}
\usepackage{IEEEtrantools}
\usepackage{array}
\usepackage{tabularx}
\pagestyle{plain}

\theoremstyle{definition}
\newtheorem{definition}{定义}

\theoremstyle{definition}
\newtheorem{example}{例}

\theoremstyle{plain}
\newtheorem{prop}{命题}

\theoremstyle{plain}
\newtheorem{theorem}{定理}

\theoremstyle{plain}
\newtheorem{lemma}{引理}

\theoremstyle{definition}
\newtheorem{property}{性质}

\newcommand{\T}{\mathsf{T}}
\newcommand{\nat}{\mathbb{N}^\star}
\newcommand{\pnat}{\mathbb{N}^\star}
\newcommand{\real}{\mathbb{R}}
\newcommand{\preal}{\mathbb{R}^{+}}
\newcommand{\integer}{\mathbb{Z}}


\begin{document}
\section*{习题3.3}
\noindent 1. 叙述函数极限$\displaystyle\lim_{n\to +\infty} f(x)$的归结原则,并应用它证明$\displaystyle\lim_{x \to +\infty} \cos \, x$不存在.

\noindent 答:设$f$为定义在正无穷邻域$[a, +\infty)$上的函数,$\displaystyle\lim_{n\to +\infty} f(x)$存在的充分必要条件是:对任何含于$[a, +\infty)$的数列$\{x_n\}$,极限$\displaystyle\lim_{n\to +\infty} f(x_n)$都存在且相等.

\begin{proof}
令$f(x) = \cos \, x$,再令$x_n = 2\pi n$,$y_n = \displaystyle\frac{\pi}{2} + 2\pi n$,则$\{x_n\}$与$\{y_n\}$都是趋于$+\infty$的数列.而
\begin{equation}
    \lim_{n \to +\infty} f(x_n) = \lim_{n \to +\infty} f(2 \pi n) = \lim_{n\to +\infty} \cos \, 2\pi n = \lim_{n \to +\infty} 1 = 1
\end{equation}
并且
\begin{equation}
    \lim_{n \to +\infty} f(y_n) = \lim_{n \to +\infty} f(\frac{\pi}{2} + 2 \pi n) = \lim_{n \to +\infty} \cos \, \left(\frac{\pi}{2}+2 \pi n\right) = \lim_{n \to +\infty} 0 = 0
\end{equation}
也就是说$\displaystyle \lim_{n\to +\infty} f(x_n) \neq \lim_{n \to +\infty} f(y_n)$,从而根据归结原则,$\displaystyle\lim_{n \to +\infty} \cos \, x$不存在.
\end{proof}

\noindent 2. 设$f$为定义在$[a, +\infty)$上的增(减)函数.证明:$\displaystyle \lim_{n \to +\infty} f(x)$存在的充要条件是$f$在$[a, +\infty)$上有上(下)界.
\begin{proof}
充分性.

\noindent 设$f$在$[a, +\infty)$上有上界,并且在$[a, +\infty)$上是增函数.设$\{ x_n \}$是一个含于$[a, +\infty)$的数列,由于$f$在$[a, +\infty)$上是增函数,所以对任意$x \in [a, +\infty)$都有$f(a) \leq f(x)$,从而对任意$n \in \nat$,都有$f(a) \leq f(x_n)$,这说明数列$\{ f(x_n) \}$有下界.设$u$是$f$在$[a, +\infty)$上的一个上界,于是对任意$x \in [a, +\infty)$,都有$f(x) \leq u$,从而对任意$n \in \nat$,都有$f(x_n) \leq u$,从而$u$也是数列$\{ f(x_n) \}$的一个上界,从而数列$\{ f(x_n) \}$有上界.显然,集合$\{ f(x_n) : n \in \nat \}$是非空的,那么根据确界原理,$\{ f(x_n) : n \in \nat \}$有唯一的上确界,设为$y_0$.现在我们来证这个$y_0$正是$f(x)$当$x \to +\infty$时候的极限.

\noindent 由于$y_0$是$\{ f(x_n) : n \in \nat \}$的上确界,于是,对任意$\epsilon > 0$,存在$f(x_\epsilon) \in \{ f(x_n) : n \in \nat \}$,使得
\begin{equation}
    y_0 - f(x_\epsilon) < \epsilon
\end{equation}
由于$\{ x_n \}$趋于$+\infty$,所以,存在$N \in \nat$,使得当$n > N$时,有$x_n > x_\epsilon$恒成立,根据函数$f$的单调性,当$x_n > x_\epsilon$时,有$f(x_n) > x_\epsilon$,但是由于$y_0$是$\{ f(x_n):n\in\nat\}$的上确界,所以$f(x_n) < y_0$,所以
\begin{equation}
    0 < y_0 - f(x_n) < y_0 - f(x_\epsilon) < \epsilon
\end{equation}
也就是
\begin{equation}
    |f(x_n) - y_0| < \epsilon
\end{equation}
这就证明了$y_0$是$f(x)$当$x \to +\infty$时候的极限,从而$f(x)$对于$x \to +\infty$的极限存在,于是充分性得证.

\noindent 必要性.
由$f$的单调性可知,对任意$x \in [a, +\infty)$,恒有$f(a) \leq f(x)$,这说明$f(a)$是$f$在$[a, +\infty)$上的一个下界,于是$f$在$[a, +\infty)$上有下界.

\noindent 设$y_0$是$f(x)$当$x \to +\infty$时候的极限.我们来证$y_0$是$f$在$[a, +\infty)$上的一个上界.采用反证法.假设$y_0$不是$f$在$[a, +\infty)$上的上界,那么必然存在$x_0 \in [a, +\infty)$,使得$f(x_0) > y_0$,设$\epsilon_0 = f(x_0) - y_0$,于是$\epsilon_0 > 0$,并且根据$f$的单调性,对所有$x > x_0$,都有$f(x) > f(x_0) > f_0$,从而$f(x) - y_0 > \epsilon_0$,但是这与$y_0$作为$f(x)$的极限的定义不符,从而矛盾.
\end{proof}

% \section*{习题6.1}
% \noindent 1. 试讨论下列函数在指定区间内是否存在一点$\epsilon$,使$f^\prime (\epsilon) = 0$;
% \begin{table}[H]
%     \centering
%     \begin{tabularx}{\textwidth} {  >{\raggedright\arraybackslash}X >{\raggedright\arraybackslash}X  }
%        (1)~~$\displaystyle f(x) = \begin{cases}
%            x \sin \, \displaystyle\frac{1}{x}, & 0 < x \leq \displaystyle\frac{1}{\pi}, \\
%            0, & x = 0;
%        \end{cases}$ & (2)~~$f(x) = |x|, \; -1 \leq x \leq 1$
%       \end{tabularx}
% \end{table}
\end{document}